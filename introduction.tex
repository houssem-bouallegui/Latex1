\chapter*{Introduction générale}
\addcontentsline{toc}{chapter}{Introduction générale} % Pour inclure l'introduction dans la table des matières
\markboth{Introduction générale}{} % Pour redéfinir l'en-tête de section

Avec l'avancement technologique et la croissance constante des entreprises, c'est devenu important de rester compétitif en mettant un effort sur l'efficacité opérationnelle et la satisfaction client. Le défi principal maintenant, c'est de sortir les produits super vite tout en assurant qu'ils sont top qualité. Et là, il faut vraiment que les équipes de développement et d’exploitation soient bien coordonnées, ainsi que la gestion des systèmes informatiques revêtent une importance cruciale.
\vskip 0.3cm

Pour notre projet de fin d'études chez l'entreprise Luceor Labs, fournisseur des équipements réseaux, nous nous serons fixés pour mission de relever ces défis au sein de l'équipe "Command & Control". Cet équipe développe et gère des logiciels qui permettent de superviser et contrôler les systèmes informatiques.
\vskip 0.3cm

L'idée, c'est d'automatiser la création d'une infrastructure pour un serveur de communication Asterisk et d'accélérer l'intégration et le déploiement continus de ce serveur tout en assurant la qualité et la sécurité du produit. Pour y arriver, nous mettrons en pratique une approche DevOps, en nous appuyant sur les services du cloud public AWS afin d'automatiser notre infrastructure en utilisant Terraform comme outil d'infrastructure en tant que code (IaC). Cette intégration Cloud permettra d'améliorer la disponibilité, l'évolutivité et l'efficacité de notre solution.
\vskip 0.3cm

Ce rapport sera structuré en quatre chapitres. 
\begin{itemize}
\item Le premier chapitre situera notre projet dans son contexte en présentant l'entreprise d'accueil, en identifiant les limites de l'existant et en proposant une solution à notre projet. 
\item Durant le deuxième chapitre, nous présenterons les concepts théoriques sous-jacents à notre travail, notamment l'approche DevOps, le Cloud Computing, l'infrastructure en tant que code, l'intégration du Cloud dans la phase de déploiement du système et l'intégration de l'IaC dans l'approvisionnement ainsi que les outils utilisés. 
\item Nous aborderons dans le troisième chapitre l'analyse des besoins et conception l'environnement du travail, les besoins fonctionnels et non fonctionnels du projet ainsi que l'architecture détaillée de la solution. 
\item Pour le dernier chapitre, nous détaillerons les phases de réalisation. Enfin, nous clôturerons par les perspectives d'amélioration et d'exploitation futures.
\end{itemize}
